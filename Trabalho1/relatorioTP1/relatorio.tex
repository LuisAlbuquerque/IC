\documentclass[11pt, oneside]{article}   	% use "amsart" instead of "article" for AMSLaTeX format
\usepackage{geometry}                		% See geometry.pdf to learn the layout options. There are lots.
\geometry{letterpaper}                   		% ... or a4paper or a5paper or ... 
%\geometry{landscape}                		% Activate for rotated page geometry
%\usepackage[parfill]{parskip}    		% Activate to begin paragraphs with an empty line rather than an indent
\usepackage{graphicx}				% Use pdf, png, jpg, or eps§ with pdflatex; use eps in DVI mode
								% TeX will automatically convert eps --> pdf in pdflatex		
\usepackage{amssymb}
\usepackage{listings}
\usepackage{imakeidx}

%SetFonts

%SetFonts


\title{Assignment 1: The rope bridge - in MCRL2}
\author{Lu\'is Albuquerque (A79010) e Rafaela Pinho (77293)}
%\date{}							% Activate to display a given date or no date

\begin{document}
\maketitle

\newpage
\tableofcontents 
\newpage

\section{Exercicio 1}
\subsection{Parâmetros do processo Aventureiro }
O Aventureiro tem duas a\c{c}\~oes possíveis. \newline
\begin{enumerate}
\item Atravessar a ponte no sentido start para finish: forward\_adventure\newline
Uma vez que o m\'aximo de aventureiros a atravessar a ponte \'e 2, \'e l\'ogico  que sempre que atravessem neste sentido v\~ao aos pares. Como a identifica\c{c}\~ao dos aventureiros \'e pela sua velocidade, forward\_adventure tem como parametro dois inteiros, que corresponde \'as suas velocidades. 
\begin{lstlisting}
forward_flashlight: Int # Int;
\end{lstlisting}

\item Atravessar a ponte no sentido finish para start:  back\_adventure\newline
O objetivo do problema \'e todos atrave\c{c}em a ponte, por isso o mais l\'ogico ser\'a apenas um aventureiro regressar para o "start". Da\'i os paramentro seja apenas um inteiro, a sua velocidade.   
\begin{lstlisting}
back_flashlight: Int;
\end{lstlisting}
\end{enumerate}

\subsection{Processo do Aventureiro }
 O processo Aventureiro recebe a sua posi\c{c}\~ao, ou "start" ou "finish" e a velocidade, que \'e o seu identificador. E caso a posi\c{c}\~ao seja "start"  avan\c{c}am dois aventureiros, ent\~ao escolhemos avan\c{c}ar com mais r\'apido dos dois, e com o outro. ?? mal explicado
 \newline Caso a posi\c{c} seja "finish" pretendemos regressar ao "start", desta vez só vai um, por isso n\~ao h\'a escolhas a fazer, é apenas regressar com a velocidade "speed" que recebemos como paramentro  e mudarmos o estado para "start".
\newline(falta acabar de explicar)
\begin{lstlisting}
proc Adventurer (pos:Position , speed:Int) =
	(pos == start ) ->
	sum s: Int .(s > speed)->
		foward_adventure(speed,s ).Adventure(finish,speed)
		<>
		foward_adventure(s,speed).Adventure(finish,speed)
	<>
	back_adventure(speed).Adventure(start,speed);
\end{lstlisting}


\section{Exercicio 2}
\subsection{Inicializar os 4 aventureiros e a lanterna}
Os aventureiros come\c{c}am todos do lado "start" e com a sua respetiva velocidade.  Para os colucar em paralelo usamos " | |".

\begin{lstlisting}
init
  Adventure(start,1) || Adventure(start,2) || Adventure(start,5) || 
  Adventure(start,10) || Flashligth(start) 
  Flashligth(start) ;
\end{lstlisting}
\subsection{Comunica\c{c}\~ao de processos}
Para que os pro\c{c}essos comuniquem usamos o comando "comm".
\begin{lstlisting}
init
 comm (
 {
  forward_adventurer | foward_adventurer | forward_flashligth -> forward, 
  back_adventurer | back_flashligth -> back	 
  },
  
  Adventure(start,1) || Adventure(start,2) || Adventure(start,5) || 
  Adventure(start,10) || Flashligth(start) 
);
\end{lstlisting}

\subsection{Garantir que apenas as a\c{c}\~oes "forward" e "back" possam ocorrer. }
Para que apenas os processos "forward" e "back" possam ocorrer usa-se o comando "allow"
\begin{lstlisting}
init
allow({ forward, back },
 comm (
 {
  forward_adventurer | foward_adventurer | forward_flashligth ->forward, 
  back_adventurer | back_flashligth -> back	 
  },
  Adventure(start,1) || Adventure(start,2) || Adventure(start,5) || 
  Adventure(start,10) || Flashligth(start) 
  )
);
\end{lstlisting}

\section{Exercicio 3}
\subsection{mcrl22lps }
\subsection{lpsxsim }

\section{Exercicio 4}
\subsection{lps2lts }
\subsection{ltsgraph }
\subsection{ltsview }

\section{Exercicio 5}
\end{enumerate}

\printindex
\end{document}  